\documentclass[]{article}
\usepackage{lmodern}
\usepackage{amssymb,amsmath}
\usepackage{ifxetex,ifluatex}
\usepackage{fixltx2e} % provides \textsubscript
\ifnum 0\ifxetex 1\fi\ifluatex 1\fi=0 % if pdftex
  \usepackage[T1]{fontenc}
  \usepackage[utf8]{inputenc}
\else % if luatex or xelatex
  \ifxetex
    \usepackage{mathspec}
  \else
    \usepackage{fontspec}
  \fi
  \defaultfontfeatures{Ligatures=TeX,Scale=MatchLowercase}
\fi
% use upquote if available, for straight quotes in verbatim environments
\IfFileExists{upquote.sty}{\usepackage{upquote}}{}
% use microtype if available
\IfFileExists{microtype.sty}{%
\usepackage{microtype}
\UseMicrotypeSet[protrusion]{basicmath} % disable protrusion for tt fonts
}{}
\usepackage[margin=1in]{geometry}
\usepackage{hyperref}
\hypersetup{unicode=true,
            pdftitle={Midterm Project},
            pdfauthor={Andrew Marshall},
            pdfborder={0 0 0},
            breaklinks=true}
\urlstyle{same}  % don't use monospace font for urls
\usepackage{color}
\usepackage{fancyvrb}
\newcommand{\VerbBar}{|}
\newcommand{\VERB}{\Verb[commandchars=\\\{\}]}
\DefineVerbatimEnvironment{Highlighting}{Verbatim}{commandchars=\\\{\}}
% Add ',fontsize=\small' for more characters per line
\usepackage{framed}
\definecolor{shadecolor}{RGB}{248,248,248}
\newenvironment{Shaded}{\begin{snugshade}}{\end{snugshade}}
\newcommand{\KeywordTok}[1]{\textcolor[rgb]{0.13,0.29,0.53}{\textbf{#1}}}
\newcommand{\DataTypeTok}[1]{\textcolor[rgb]{0.13,0.29,0.53}{#1}}
\newcommand{\DecValTok}[1]{\textcolor[rgb]{0.00,0.00,0.81}{#1}}
\newcommand{\BaseNTok}[1]{\textcolor[rgb]{0.00,0.00,0.81}{#1}}
\newcommand{\FloatTok}[1]{\textcolor[rgb]{0.00,0.00,0.81}{#1}}
\newcommand{\ConstantTok}[1]{\textcolor[rgb]{0.00,0.00,0.00}{#1}}
\newcommand{\CharTok}[1]{\textcolor[rgb]{0.31,0.60,0.02}{#1}}
\newcommand{\SpecialCharTok}[1]{\textcolor[rgb]{0.00,0.00,0.00}{#1}}
\newcommand{\StringTok}[1]{\textcolor[rgb]{0.31,0.60,0.02}{#1}}
\newcommand{\VerbatimStringTok}[1]{\textcolor[rgb]{0.31,0.60,0.02}{#1}}
\newcommand{\SpecialStringTok}[1]{\textcolor[rgb]{0.31,0.60,0.02}{#1}}
\newcommand{\ImportTok}[1]{#1}
\newcommand{\CommentTok}[1]{\textcolor[rgb]{0.56,0.35,0.01}{\textit{#1}}}
\newcommand{\DocumentationTok}[1]{\textcolor[rgb]{0.56,0.35,0.01}{\textbf{\textit{#1}}}}
\newcommand{\AnnotationTok}[1]{\textcolor[rgb]{0.56,0.35,0.01}{\textbf{\textit{#1}}}}
\newcommand{\CommentVarTok}[1]{\textcolor[rgb]{0.56,0.35,0.01}{\textbf{\textit{#1}}}}
\newcommand{\OtherTok}[1]{\textcolor[rgb]{0.56,0.35,0.01}{#1}}
\newcommand{\FunctionTok}[1]{\textcolor[rgb]{0.00,0.00,0.00}{#1}}
\newcommand{\VariableTok}[1]{\textcolor[rgb]{0.00,0.00,0.00}{#1}}
\newcommand{\ControlFlowTok}[1]{\textcolor[rgb]{0.13,0.29,0.53}{\textbf{#1}}}
\newcommand{\OperatorTok}[1]{\textcolor[rgb]{0.81,0.36,0.00}{\textbf{#1}}}
\newcommand{\BuiltInTok}[1]{#1}
\newcommand{\ExtensionTok}[1]{#1}
\newcommand{\PreprocessorTok}[1]{\textcolor[rgb]{0.56,0.35,0.01}{\textit{#1}}}
\newcommand{\AttributeTok}[1]{\textcolor[rgb]{0.77,0.63,0.00}{#1}}
\newcommand{\RegionMarkerTok}[1]{#1}
\newcommand{\InformationTok}[1]{\textcolor[rgb]{0.56,0.35,0.01}{\textbf{\textit{#1}}}}
\newcommand{\WarningTok}[1]{\textcolor[rgb]{0.56,0.35,0.01}{\textbf{\textit{#1}}}}
\newcommand{\AlertTok}[1]{\textcolor[rgb]{0.94,0.16,0.16}{#1}}
\newcommand{\ErrorTok}[1]{\textcolor[rgb]{0.64,0.00,0.00}{\textbf{#1}}}
\newcommand{\NormalTok}[1]{#1}
\usepackage{graphicx,grffile}
\makeatletter
\def\maxwidth{\ifdim\Gin@nat@width>\linewidth\linewidth\else\Gin@nat@width\fi}
\def\maxheight{\ifdim\Gin@nat@height>\textheight\textheight\else\Gin@nat@height\fi}
\makeatother
% Scale images if necessary, so that they will not overflow the page
% margins by default, and it is still possible to overwrite the defaults
% using explicit options in \includegraphics[width, height, ...]{}
\setkeys{Gin}{width=\maxwidth,height=\maxheight,keepaspectratio}
\IfFileExists{parskip.sty}{%
\usepackage{parskip}
}{% else
\setlength{\parindent}{0pt}
\setlength{\parskip}{6pt plus 2pt minus 1pt}
}
\setlength{\emergencystretch}{3em}  % prevent overfull lines
\providecommand{\tightlist}{%
  \setlength{\itemsep}{0pt}\setlength{\parskip}{0pt}}
\setcounter{secnumdepth}{0}
% Redefines (sub)paragraphs to behave more like sections
\ifx\paragraph\undefined\else
\let\oldparagraph\paragraph
\renewcommand{\paragraph}[1]{\oldparagraph{#1}\mbox{}}
\fi
\ifx\subparagraph\undefined\else
\let\oldsubparagraph\subparagraph
\renewcommand{\subparagraph}[1]{\oldsubparagraph{#1}\mbox{}}
\fi

%%% Use protect on footnotes to avoid problems with footnotes in titles
\let\rmarkdownfootnote\footnote%
\def\footnote{\protect\rmarkdownfootnote}

%%% Change title format to be more compact
\usepackage{titling}

% Create subtitle command for use in maketitle
\newcommand{\subtitle}[1]{
  \posttitle{
    \begin{center}\large#1\end{center}
    }
}

\setlength{\droptitle}{-2em}
  \title{Midterm Project}
  \pretitle{\vspace{\droptitle}\centering\huge}
  \posttitle{\par}
  \author{Andrew Marshall}
  \preauthor{\centering\large\emph}
  \postauthor{\par}
  \predate{\centering\large\emph}
  \postdate{\par}
  \date{July 17, 2018}


\begin{document}
\maketitle

\subsection{R Markdown}\label{r-markdown}

This is an R Markdown document. Markdown is a simple formatting syntax
for authoring HTML, PDF, and MS Word documents. For more details on
using R Markdown see \url{http://rmarkdown.rstudio.com}.

When you click the \textbf{Knit} button a document will be generated
that includes both content as well as the output of any embedded R code
chunks within the document. You can embed an R code chunk like this:

\begin{Shaded}
\begin{Highlighting}[]
\CommentTok{#This will generate a data frame that includes data from the basic "Eggs in a Nest" recipe circa 1936, with the NDB_No included.  The matches were done manually with a combination of single match and picking the best one.}
\NormalTok{Eggs_in_a_Nest1936 <-}\StringTok{ }\KeywordTok{data.frame}\NormalTok{(}
  \DataTypeTok{NDB_No =} \KeywordTok{c}\NormalTok{(}\DecValTok{11371}\NormalTok{,}\DecValTok{01077}\NormalTok{,}\DecValTok{07029}\NormalTok{,}\DecValTok{01123}\NormalTok{,}\DecValTok{18069}\NormalTok{,}\DecValTok{01001}\NormalTok{),}
  \DataTypeTok{Amount =} \KeywordTok{c}\NormalTok{(}\FloatTok{2.00}\NormalTok{,}\FloatTok{5.00}\NormalTok{,}\FloatTok{2.67}\NormalTok{,}\FloatTok{4.00}\NormalTok{,}\FloatTok{0.25}\NormalTok{,}\FloatTok{2.00}\NormalTok{),}
  \DataTypeTok{Measure =} \KeywordTok{c}\NormalTok{(}\StringTok{"cups"}\NormalTok{,}\StringTok{"Tbs"}\NormalTok{,}\StringTok{"oz"}\NormalTok{,}\StringTok{"lrg"}\NormalTok{,}\StringTok{"cup"}\NormalTok{,}\StringTok{"Tbs"}\NormalTok{),}
  \DataTypeTok{Ingredients =} \KeywordTok{c}\NormalTok{(}\StringTok{"Potatoes, mashed, home-prepared, whole milk and margarine added"}\NormalTok{,}\StringTok{"Milk, whole, 3.25% milkfat, without added vitamin A and vitamin D"}\NormalTok{,}\StringTok{"Ham, sliced, regular (approximately 11% fat)"}\NormalTok{,}\StringTok{"Egg, whole, raw, fresh"}\NormalTok{,}\StringTok{"Bread, white, commercially prepared (includes soft bread crumbs)"}\NormalTok{,}\StringTok{"Butter, salted"}\NormalTok{)}
\NormalTok{)}

\CommentTok{#This will generate a data frame that includes data from the basic "Eggs in a Nest" recipe circa 2006, with the NDB_No included.  The matches were done manually with a combination of single match and picking the best one.}
\NormalTok{Eggs_in_a_Nest2006 <-}\StringTok{ }\KeywordTok{data.frame}\NormalTok{(}
  \DataTypeTok{NDB_No =} \KeywordTok{c}\NormalTok{(}\DecValTok{11371}\NormalTok{,}\DecValTok{01077}\NormalTok{,}\DecValTok{07029}\NormalTok{,}\DecValTok{01123}\NormalTok{,}\DecValTok{18069}\NormalTok{,}\DecValTok{01001}\NormalTok{),}
  \DataTypeTok{Amount =} \KeywordTok{c}\NormalTok{(}\FloatTok{2.00}\NormalTok{,}\FloatTok{5.00}\NormalTok{,}\FloatTok{2.67}\NormalTok{,}\FloatTok{4.00}\NormalTok{,}\FloatTok{0.25}\NormalTok{,}\FloatTok{2.00}\NormalTok{),}
  \DataTypeTok{Measure =} \KeywordTok{c}\NormalTok{(}\StringTok{"cups"}\NormalTok{,}\StringTok{"Tbs"}\NormalTok{,}\StringTok{"oz"}\NormalTok{,}\StringTok{"lrg"}\NormalTok{,}\StringTok{"cup"}\NormalTok{,}\StringTok{"Tbs"}\NormalTok{),}
  \DataTypeTok{Ingredients =} \KeywordTok{c}\NormalTok{(}\StringTok{"Potatoes, mashed, home-prepared, whole milk and margarine added"}\NormalTok{,}\StringTok{"Milk, whole, 3.25% milkfat, without added vitamin A and vitamin D"}\NormalTok{,}\StringTok{"Ham, sliced, regular (approximately 11% fat)"}\NormalTok{,}\StringTok{"Egg, whole, raw, fresh"}\NormalTok{,}\StringTok{"Bread, white, commercially prepared (includes soft bread crumbs)"}\NormalTok{,}\StringTok{"Butter, salted"}\NormalTok{)}
\NormalTok{  )}


\CommentTok{#This will generate a data frame that includes data from the basic "Scalloped Potatoes" recipe circa 1936, with the NDB_No included.  The matches were done manually with a combination of single match and picking the best one. The original recipe does not specify what kind of grated cheese to use, so Parmesan is being substitued in because there is not a shredded cheddar listing provided in the FOOD_DES.txt file.}

\NormalTok{Scalloped_Potatoes1936 <-}\StringTok{ }\KeywordTok{data.frame}\NormalTok{(}
  \DataTypeTok{NDB_No =} \KeywordTok{c}\NormalTok{(}\DecValTok{11354}\NormalTok{,}\DecValTok{01001}\NormalTok{,}\DecValTok{20082}\NormalTok{,}\DecValTok{01077}\NormalTok{,}\DecValTok{01032}\NormalTok{,}\DecValTok{11943}\NormalTok{),}
  \DataTypeTok{Amount =} \KeywordTok{c}\NormalTok{(}\FloatTok{4.00}\NormalTok{,}\FloatTok{3.00}\NormalTok{,}\FloatTok{3.00}\NormalTok{,}\FloatTok{1.50}\NormalTok{,}\FloatTok{0.25}\NormalTok{,}\FloatTok{0.75}\NormalTok{),}
  \DataTypeTok{Measure =} \KeywordTok{c}\NormalTok{(}\StringTok{"cups"}\NormalTok{,}\StringTok{"Tbs"}\NormalTok{,}\StringTok{"Tbs"}\NormalTok{,}\StringTok{"cups"}\NormalTok{,}\StringTok{"lb"}\NormalTok{,}\StringTok{"cup"}\NormalTok{),}
  \DataTypeTok{Ingredients =} \KeywordTok{c}\NormalTok{(}\StringTok{"Potatoes, white, flesh and skin, raw"}\NormalTok{,}\StringTok{"Butter, salted"}\NormalTok{,}\StringTok{"Wheat flour, white, all-purpose, self-rising, enriched"}\NormalTok{,}\StringTok{"Milk, whole, 3.25% milkfat, with added vitamin D"}\NormalTok{,}\StringTok{"Cheese, parmesan, grated"}\NormalTok{,}\StringTok{"Pimento, canned"}\NormalTok{)}

\NormalTok{)}

\CommentTok{#This will generate a data frame that includes data from the basic "Scalloped Potatoes" recipe circa 2006, with the NDB_No included.  The matches were done manually with a combination of single match and picking the best one. The recipe calls for 0.5 cup of grated Parmesan instead of the 0.25 cups of shredded cheddar in the esha research document, so the Parmesan is being substituted in. Also, there is not a shredded cheddar listing provided in the FOOD_DES.txt file.}

\NormalTok{Scalloped_Potatoest2006 <-}\StringTok{ }\KeywordTok{data.frame}\NormalTok{(}
  \DataTypeTok{NDB_No =} \KeywordTok{c}\NormalTok{(}\DecValTok{11354}\NormalTok{,}\DecValTok{06043}\NormalTok{,}\DecValTok{01077}\NormalTok{,}\DecValTok{01032}\NormalTok{),}
  \DataTypeTok{Amount =} \KeywordTok{c}\NormalTok{(}\FloatTok{3.00}\NormalTok{,}\FloatTok{10.75}\NormalTok{,}\FloatTok{1.50}\NormalTok{,}\FloatTok{0.50}\NormalTok{),}
  \DataTypeTok{Measure =} \KeywordTok{c}\NormalTok{(}\StringTok{"cups"}\NormalTok{,}\StringTok{"oz"}\NormalTok{,}\StringTok{"cups"}\NormalTok{,}\StringTok{"cup"}\NormalTok{),}
   \DataTypeTok{Ingredients =} \KeywordTok{c}\NormalTok{(}\StringTok{"Potatoes, white, flesh and skin, raw"}\NormalTok{,}\StringTok{"Soup, cream of mushroom, canned, condensed"}\NormalTok{,}\StringTok{"Milk, whole, 3.25% milkfat, with added vitamin D"}\NormalTok{,}\StringTok{"Cheese, parmesan, grated"}\NormalTok{)}
\NormalTok{)}
\end{Highlighting}
\end{Shaded}

\begin{Shaded}
\begin{Highlighting}[]
\CommentTok{#This following code will export the previously created data frames to individual tab delimited files.}

\CommentTok{#The data frame for the 1936 is being written to a tab-delimited file.}
\KeywordTok{write.table}\NormalTok{(Eggs_in_a_Nest1936, }\StringTok{"Eggs_in_a_Nest1936.tab"}\NormalTok{, }\DataTypeTok{sep=}\StringTok{"}\CharTok{\textbackslash{}t}\StringTok{"}\NormalTok{,}\DataTypeTok{row.names=}\NormalTok{F)}

\CommentTok{#The data frame for the 2006 is being written to a tab-delimited file.}
\KeywordTok{write.table}\NormalTok{(Eggs_in_a_Nest2006, }\StringTok{"Eggs_in_a_Nest2006.tab"}\NormalTok{, }\DataTypeTok{sep=}\StringTok{"}\CharTok{\textbackslash{}t}\StringTok{"}\NormalTok{,}\DataTypeTok{row.names=}\NormalTok{F)}

\CommentTok{#The data frame for the 1936 is being written to a tab-delimited file.}
\KeywordTok{write.table}\NormalTok{(Scalloped_Potatoes1936, }\StringTok{"Scalloped_Potatoes1936.tab"}\NormalTok{, }\DataTypeTok{sep=}\StringTok{"}\CharTok{\textbackslash{}t}\StringTok{"}\NormalTok{,}\DataTypeTok{row.names=}\NormalTok{F)}

\CommentTok{#The data frame for the 2006 is being written to a tab-delimited file.}
\KeywordTok{write.table}\NormalTok{(Scalloped_Potatoest2006, }\StringTok{"Scalloped_Potatoes2006.tab"}\NormalTok{, }\DataTypeTok{sep=}\StringTok{"}\CharTok{\textbackslash{}t}\StringTok{"}\NormalTok{,}\DataTypeTok{row.names=}\NormalTok{F)}
\end{Highlighting}
\end{Shaded}

\begin{Shaded}
\begin{Highlighting}[]
\CommentTok{#This following code will read and import the 2 files created above to verify they were created correctly and then the two files will be merged}

\CommentTok{#Assigning path to TAB file to variable PathToEIN1936}
\NormalTok{PathToEIN1936 =}\StringTok{ "C:/Users/drewm/Documents/GitHub/code-stat700/MidTerm Project/Eggs_in_a_Nest1936.tab"}

\CommentTok{#Assigning data from TAB file to data frame}
\NormalTok{Eggs_in_a_Nest1936.df <-}\StringTok{ }\KeywordTok{read.delim}\NormalTok{(PathToEIN1936,}\DataTypeTok{header=}\OtherTok{TRUE}\NormalTok{,}\DataTypeTok{skip=} \DecValTok{0}\NormalTok{,}\DataTypeTok{sep =} \StringTok{"}\CharTok{\textbackslash{}t}\StringTok{"}\NormalTok{,}\DataTypeTok{as.is=}\OtherTok{TRUE}\NormalTok{)}

\CommentTok{#Assigning path to TAB file to variable PathTo2006}
\NormalTok{PathToEIN2006 =}\StringTok{ "C:/Users/drewm/Documents/GitHub/code-stat700/MidTerm Project/Eggs_in_a_Nest2006.tab"}

\CommentTok{#Assigning data from CSV file to data frame}
\NormalTok{Eggs_in_a_Nest2006.df <-}\StringTok{ }\KeywordTok{read.delim}\NormalTok{(PathToEIN2006,}\DataTypeTok{header=}\OtherTok{TRUE}\NormalTok{,}\DataTypeTok{skip=} \DecValTok{0}\NormalTok{,}\DataTypeTok{sep =} \StringTok{"}\CharTok{\textbackslash{}t}\StringTok{"}\NormalTok{,}\DataTypeTok{as.is=}\OtherTok{TRUE}\NormalTok{)}

\CommentTok{#Now that the files have been verified, it is time to merge the 2 data frames by Year.}
\NormalTok{Eggs_in_a_Nest1936_}\DecValTok{2006}\NormalTok{ <-}\StringTok{ }\KeywordTok{merge.data.frame}\NormalTok{(Eggs_in_a_Nest1936.df,Eggs_in_a_Nest2006.df,}\DataTypeTok{by.x =} \StringTok{"Ingredients"}\NormalTok{,}\DataTypeTok{by.y =} \StringTok{"Ingredients"}\NormalTok{)}
\KeywordTok{colnames}\NormalTok{(Eggs_in_a_Nest1936_}\DecValTok{2006}\NormalTok{) <-}\StringTok{ }\KeywordTok{c}\NormalTok{(}\StringTok{"Ingredients"}\NormalTok{,}\StringTok{"Amount1936"}\NormalTok{,}\StringTok{"Measure1936"}\NormalTok{,}\StringTok{"Amount2006"}\NormalTok{,}\StringTok{"Measure2006"}\NormalTok{)}

\CommentTok{#Assigning path to TAB file to variable PathTo1936}
\NormalTok{PathToSP1936 =}\StringTok{ "C:/Users/drewm/Documents/GitHub/code-stat700/MidTerm Project/Scalloped_Potatoes1936.tab"}

\CommentTok{#Assigning data from TAB file to data frame}
\NormalTok{Scalloped_Potatoes1936.df <-}\StringTok{ }\KeywordTok{read.delim}\NormalTok{(PathToSP1936,}\DataTypeTok{header=}\OtherTok{TRUE}\NormalTok{,}\DataTypeTok{skip=} \DecValTok{0}\NormalTok{,}\DataTypeTok{sep =} \StringTok{"}\CharTok{\textbackslash{}t}\StringTok{"}\NormalTok{,}\DataTypeTok{as.is=}\OtherTok{TRUE}\NormalTok{)}

\CommentTok{#Assigning path to TAB file to variable PathTo2006}
\NormalTok{PathToSP2006 =}\StringTok{ "C:/Users/drewm/Documents/GitHub/code-stat700/MidTerm Project/Scalloped_Potatoes2006.tab"}

\CommentTok{#Assigning data from CSV file to data frame}
\NormalTok{Scalloped_Potatoes2006.df <-}\StringTok{ }\KeywordTok{read.delim}\NormalTok{(PathToSP2006,}\DataTypeTok{header=}\OtherTok{TRUE}\NormalTok{,}\DataTypeTok{skip=} \DecValTok{0}\NormalTok{,}\DataTypeTok{sep =} \StringTok{"}\CharTok{\textbackslash{}t}\StringTok{"}\NormalTok{,}\DataTypeTok{as.is=}\OtherTok{TRUE}\NormalTok{)}

\CommentTok{#Now that the files have been verified, it is time to merge the 2 data frames by Year.}
\NormalTok{Scalloped_Potatoes1936_}\DecValTok{2006}\NormalTok{ <-}\StringTok{ }\KeywordTok{merge.data.frame}\NormalTok{(Scalloped_Potatoes1936.df,Scalloped_Potatoes2006.df,}\DataTypeTok{by.x =} \StringTok{"Ingredients"}\NormalTok{,}\DataTypeTok{by.y =} \StringTok{"Ingredients"}\NormalTok{)}
\KeywordTok{colnames}\NormalTok{(Scalloped_Potatoes1936_}\DecValTok{2006}\NormalTok{) <-}\StringTok{ }\KeywordTok{c}\NormalTok{(}\StringTok{"Ingredients"}\NormalTok{,}\StringTok{"Amount1936"}\NormalTok{,}\StringTok{"Measure1936"}\NormalTok{,}\StringTok{"Amount2006"}\NormalTok{,}\StringTok{"Measure2006"}\NormalTok{)}
\end{Highlighting}
\end{Shaded}

\begin{Shaded}
\begin{Highlighting}[]
\CommentTok{#This section will verify that the data in both the 1936 and 2006 data sets match. If there is a match then the match function will return the vector of the position of the first vector in the second vector;Otherwise an '0' result has been set to return if there is no match. No match is a possible indicator that different ingredients were used at the time each recipe was created.}

\KeywordTok{match}\NormalTok{(Eggs_in_a_Nest1936.df}\OperatorTok{$}\NormalTok{Ingredients,Eggs_in_a_Nest2006.df}\OperatorTok{$}\NormalTok{Ingredients,}\DataTypeTok{nomatch =} \DecValTok{0}\NormalTok{)}
\end{Highlighting}
\end{Shaded}

\begin{verbatim}
## [1] 1 2 3 4 5 6
\end{verbatim}

\begin{Shaded}
\begin{Highlighting}[]
\KeywordTok{match}\NormalTok{(Eggs_in_a_Nest1936.df}\OperatorTok{$}\NormalTok{Amount,Eggs_in_a_Nest2006.df}\OperatorTok{$}\NormalTok{Amount,}\DataTypeTok{nomatch =} \DecValTok{0}\NormalTok{)}
\end{Highlighting}
\end{Shaded}

\begin{verbatim}
## [1] 1 2 3 4 5 1
\end{verbatim}

\begin{Shaded}
\begin{Highlighting}[]
\KeywordTok{match}\NormalTok{(Eggs_in_a_Nest1936.df}\OperatorTok{$}\NormalTok{Measure,Eggs_in_a_Nest2006.df}\OperatorTok{$}\NormalTok{Measure,}\DataTypeTok{nomatch =} \DecValTok{0}\NormalTok{)}
\end{Highlighting}
\end{Shaded}

\begin{verbatim}
## [1] 1 2 3 4 5 2
\end{verbatim}

\begin{Shaded}
\begin{Highlighting}[]
\KeywordTok{match}\NormalTok{(Scalloped_Potatoes1936.df}\OperatorTok{$}\NormalTok{Ingredients,Scalloped_Potatoes2006.df}\OperatorTok{$}\NormalTok{Ingredients,}\DataTypeTok{nomatch =} \DecValTok{0}\NormalTok{)}
\end{Highlighting}
\end{Shaded}

\begin{verbatim}
## [1] 1 0 0 3 4 0
\end{verbatim}

\begin{Shaded}
\begin{Highlighting}[]
\KeywordTok{match}\NormalTok{(Scalloped_Potatoes1936.df}\OperatorTok{$}\NormalTok{Amount,Scalloped_Potatoes2006.df}\OperatorTok{$}\NormalTok{Amount,}\DataTypeTok{nomatch =} \DecValTok{0}\NormalTok{)}
\end{Highlighting}
\end{Shaded}

\begin{verbatim}
## [1] 0 1 1 3 0 0
\end{verbatim}

\begin{Shaded}
\begin{Highlighting}[]
\KeywordTok{match}\NormalTok{(Scalloped_Potatoes1936.df}\OperatorTok{$}\NormalTok{Measure,Scalloped_Potatoes2006.df}\OperatorTok{$}\NormalTok{Measure,}\DataTypeTok{nomatch =} \DecValTok{0}\NormalTok{)}
\end{Highlighting}
\end{Shaded}

\begin{verbatim}
## [1] 1 0 0 1 0 4
\end{verbatim}

\begin{Shaded}
\begin{Highlighting}[]
\CommentTok{#This section will import the Recipes.csv file and once imported the recipe data will be appended the existing data in the file;After the append is complete, the Recipes.csv file will be exported out with the new data.}

\NormalTok{Recipes_csv_import <-}\StringTok{ }\KeywordTok{read.csv}\NormalTok{(}\DataTypeTok{file=}\StringTok{"C:/Users/drewm/Documents/GitHub/code-stat700/MidTerm Project/Recipes.csv"}\NormalTok{, }\DataTypeTok{header=}\OtherTok{TRUE}\NormalTok{, }\DataTypeTok{sep=}\StringTok{","}\NormalTok{)}

\NormalTok{Eggs_in_a_Nest1936_import <-}\StringTok{ }\KeywordTok{data.frame}\NormalTok{(}\KeywordTok{matrix}\NormalTok{(}\KeywordTok{c}\NormalTok{(}\StringTok{"Eggs in a Nest"}\NormalTok{,}\StringTok{"1936"}\NormalTok{,}\StringTok{"4"}\NormalTok{,}\StringTok{"4"}\NormalTok{),}\DataTypeTok{nrow =} \DecValTok{1}\NormalTok{,}\DataTypeTok{ncol =} \DecValTok{4}\NormalTok{))}
\KeywordTok{colnames}\NormalTok{(Eggs_in_a_Nest1936_import) <-}\StringTok{ }\KeywordTok{c}\NormalTok{(}\StringTok{"Recipe"}\NormalTok{,}\StringTok{"Year"}\NormalTok{,}\StringTok{"MinServings"}\NormalTok{,}\StringTok{"MaxServings"}\NormalTok{)}

\NormalTok{Recipes_csv_exportEIN <-}\StringTok{ }\KeywordTok{rbind.data.frame}\NormalTok{(Recipes_csv_import,Eggs_in_a_Nest1936_import)}

\NormalTok{Eggs_in_a_Nest2006_import <-}\StringTok{ }\KeywordTok{data.frame}\NormalTok{(}\KeywordTok{matrix}\NormalTok{(}\KeywordTok{c}\NormalTok{(}\StringTok{"Eggs in a Nest"}\NormalTok{,}\StringTok{"2006"}\NormalTok{,}\StringTok{"4"}\NormalTok{,}\StringTok{"4"}\NormalTok{),}\DataTypeTok{nrow =} \DecValTok{1}\NormalTok{,}\DataTypeTok{ncol =} \DecValTok{4}\NormalTok{))}
\KeywordTok{colnames}\NormalTok{(Eggs_in_a_Nest2006_import) <-}\StringTok{ }\KeywordTok{c}\NormalTok{(}\StringTok{"Recipe"}\NormalTok{,}\StringTok{"Year"}\NormalTok{,}\StringTok{"MinServings"}\NormalTok{,}\StringTok{"MaxServings"}\NormalTok{)}

\NormalTok{Recipes_csv_exportEIN <-}\StringTok{ }\KeywordTok{rbind.data.frame}\NormalTok{(Recipes_csv_exportEIN,Eggs_in_a_Nest2006_import)}

\CommentTok{#The Recipes file will now be exported out to as CSV again with its new data}
\KeywordTok{write.table}\NormalTok{(Recipes_csv_exportEIN, }\StringTok{"Recipes.csv"}\NormalTok{, }\DataTypeTok{sep=}\StringTok{","}\NormalTok{,}\DataTypeTok{row.names=}\NormalTok{F)}

\NormalTok{Scalloped_Potatoes1936_import <-}\StringTok{ }\KeywordTok{data.frame}\NormalTok{(}\KeywordTok{matrix}\NormalTok{(}\KeywordTok{c}\NormalTok{(}\StringTok{"Scalloped Potatoes"}\NormalTok{,}\StringTok{"1936"}\NormalTok{,}\StringTok{"8"}\NormalTok{,}\StringTok{"8"}\NormalTok{),}\DataTypeTok{nrow =} \DecValTok{1}\NormalTok{,}\DataTypeTok{ncol =} \DecValTok{4}\NormalTok{))}
\KeywordTok{colnames}\NormalTok{(Scalloped_Potatoes1936_import) <-}\StringTok{ }\KeywordTok{c}\NormalTok{(}\StringTok{"Recipe"}\NormalTok{,}\StringTok{"Year"}\NormalTok{,}\StringTok{"MinServings"}\NormalTok{,}\StringTok{"MaxServings"}\NormalTok{)}

\NormalTok{Recipes_csv_exportSP <-}\StringTok{ }\KeywordTok{rbind.data.frame}\NormalTok{(Recipes_csv_exportEIN,Scalloped_Potatoes1936_import)}

\NormalTok{Scalloped_Potatoes2006_import <-}\StringTok{ }\KeywordTok{data.frame}\NormalTok{(}\KeywordTok{matrix}\NormalTok{(}\KeywordTok{c}\NormalTok{(}\StringTok{"Scalloped Potatoes"}\NormalTok{,}\StringTok{"2006"}\NormalTok{,}\StringTok{"6"}\NormalTok{,}\StringTok{"6"}\NormalTok{),}\DataTypeTok{nrow =} \DecValTok{1}\NormalTok{,}\DataTypeTok{ncol =} \DecValTok{4}\NormalTok{))}
\KeywordTok{colnames}\NormalTok{(Scalloped_Potatoes2006_import) <-}\StringTok{ }\KeywordTok{c}\NormalTok{(}\StringTok{"Recipe"}\NormalTok{,}\StringTok{"Year"}\NormalTok{,}\StringTok{"MinServings"}\NormalTok{,}\StringTok{"MaxServings"}\NormalTok{)}

\NormalTok{Recipes_csv_exportSP <-}\StringTok{ }\KeywordTok{rbind.data.frame}\NormalTok{(Recipes_csv_exportSP,Scalloped_Potatoes2006_import)}

\CommentTok{#The Recipes file will now be exported out to as CSV again with its new data}
\KeywordTok{write.table}\NormalTok{(Recipes_csv_exportSP, }\StringTok{"Recipes.csv"}\NormalTok{, }\DataTypeTok{sep=}\StringTok{","}\NormalTok{,}\DataTypeTok{row.names=}\NormalTok{F)}
\end{Highlighting}
\end{Shaded}


\end{document}
